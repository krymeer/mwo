\documentclass[a4paper,11pt]{article}
\usepackage{geometry}
 \geometry{
  a4paper,
  total={160mm,247mm},
  left=25mm,
  top=25mm,
 }
\usepackage[T1]{fontenc}
\usepackage[polish]{babel}
\usepackage[utf8]{inputenc}
\usepackage{fancyhdr, lipsum, setspace, amsmath, array, multirow, color, colortbl, listings, graphicx, MnSymbol, xcolor, float, multicol, ragged2e, algorithm2e, hyperref}

\pagestyle{fancy}
\fancyhf{}
\lhead{R. Makowski, K. Osada, M. Regulski, K. Tatarynowicz}
\rhead{Dokument projektowy}
\cfoot{\thepage}
\setlength{\headheight}{14pt}

\hypersetup{
  colorlinks  = true,
  linkcolor   = black,
  citecolor   = black,
  urlcolor    = blue
}
\urlstyle{same}

\onehalfspacing
\frenchspacing
\author{Remigiusz Makowski, Krzysztof Osada, Marcin Regulski, Krzysztof Tatarynowicz}
\title{\textbf{\Huge{Metody wytwarzania oprogramowania}}\\[3pt] Dokument projektowy}
\begin{document}
  \singlespacing
    \maketitle\thispagestyle{empty}
  \onehalfspacing
  \section{Temat realizowanego projektu}
    Aplikacja „TO-DO” oferująca (w podstawowej wersji) dodawanie, usuwanie i modyfikowanie notatek .
  
  \section{Skład grupy projektowej}
  \begin{enumerate}
    \item Remigiusz Makowski 
    \item Krzysztof Osada 
    \item Marcin Regulski 
    \item Krzysztof Tatarynowicz 
  \end{enumerate}
    
  \section{Specyfikacja wymagań}
  \begin{enumerate}
    \item System umożliwia użytkownikom zarządzanie notatkami ,,To-Do''.
    \item Użytkownik ma możliwość dodawania notatek do własnej listy, edycji oraz usuwania notatek. 
    \item Użytkownik uzyskuje dostęp do systemu poprzez dowolną przeglądarkę internetową. 
    \item System zapewnia użytkownikowi łatwy, wygodny i nowoczesny interfejs graficzny -- niezależnie od wybranej przeglądarki. Układ ten jest w pełni responsywny. 
    \item System zapewnia przejrzysty i poprawnie działający interfejs na urządzeniach mobilnych. 
    \item System gwarantuje bezpieczeństwo notatek użytkownika. 
    \item System posiada mechanizm rozpoznawania użytkowników. Mechanizm ten opiera się na unikalnych identyfikatorach oraz haśle dostępu. 
    \item System umożliwia synchronizację danych z wybranymi serwisami społecznościowymi (np. Google Calendar, Facebook).
    \item System działa w oparciu o nowe technologie. 
    \item System posiada możliwość łatwej rozbudowy o dodatkowe komponenty.
  \end{enumerate}

  \section{WBS} 
  Znajduje się pod adresem:\newline
  \url{https://raw.githubusercontent.com/krymeer/mwo/master/zadania/2/WBS/WBS.png}.

  \section{Wstępny harmonogram projektu}
  \begin{enumerate}
    \item Ramowy podział zadań -- do 24 października. 
    \item Zapewnienie działania podstawowych funkcjonalności (front-end, back-end, dodawanie, usuwanie, modyfikowanie notatek itd.) -- do 13 listopada. 
    \item Dołączenie ewentualnych rozszerzeń -- do 27 listopada. 
    \item Przetestowanie działania aplikacji -- do 30 listopada. 
    \item Zakończenie projektu zespołowego -- do 8 grudnia.
  \end{enumerate}

  \section{Cechy charakterystyczne wybranych technologii}
  \begin{enumerate}
    \item CSS, HTML -- języki wykorzystywane do budowania statycznych stron internetowych. Pierwszy z nich służy do tworzenia list dyrektyw i reguł określających, jak ma być wyświetlana dana witryna, drugi zaś odpowiada za podział strony na mniejsze części („szkielet”). 
    \item CSS Grid Layout -- moduł CSS3 umożliwiający swobodne dzielenie bloków strony internetowej na mniejsze, prostokątne fragmenty. Jego innowacyjność bierze się stąd, że dotychczas taki podział był możliwy jedynie w jednym wymiarze. 
    \item Vue.js -- kompaktowy framework upraszczający budowanie interfejsów aplikacji webowych, ze szczególnym naciskiem na reaktywność interfejsu. W przeciwieństwie do głównych konkurentów -- Angulara i Reacta -- Vue nie narzuca całej struktury aplikacji. Ma też znacznie mniejsze API, dzięki czemu jest łatwiejszy do przyswojenia. 
    \item ServiceWorker -- skrypt uruchamiany przez przeglądarkę w tle (w odróżnieniu od strony internetowej). Działa jako proxy pomiędzy stroną a przeglądarką i siecią, pozwalając m.in. na tworzenie aplikacji działających w trybie offline. 
    \item DynamoDB -- nierelacyjna baza danych zapewniająca dostęp do informacji w bardzo krótkim, milisekundowym czasie  w każdym rozmiarze. Co więcej, ma wiele zautomatyzowanych funkcji i jest elastyczna.
  \end{enumerate}

  \section{Uzasadnienie, dlaczego dana technologia powinna znaleźć się w\,projekcie}
  \begin{enumerate}
    \item CSS, HTML -- podstawa nowoczesnych stron WWW.
    \item CSS Grid Layout -- jest nowoczesnym modułem i znacząco ułatwia dzielenie stron na mniejsze fragmenty.
    \item Vue.js pozwala w prosty i uporządkowany sposób zarządzać przepływem informacji w interfejsie użytkownika. W porównaniu do innych frameworków jest lżejszy i prostszy w obsłudze; wartość dodana (w stosunku do nieużywania frameworka) to zapewnienie jednolitego mechanizmu obsługi reakcji interfejsu na zmiany w danych (data binding).
    \item ServiceWorker umożliwi korzystanie z aplikacji nawet wtedy, kiedy sieć internetowa nie jest w ogóle dostępna. Taka funkcjonalność może stanowić jedną z cech wyróżniających projekt. 
    \item DynamoDB - szybkość działania oraz kontrola poprzez komendy w języku JavaScript, z\,którego projekt w dużej mierze korzysta. Mniejszy początkowy narzut pracy w stosunku do tradycyjnych baz relacyjnych. 
  \end{enumerate}
\end{document}